\documentclass{article}
\usepackage[utf8]{inputenc}


\begin{document}

Dear Editors,

\vspace{2em}

Thank you for coming back to us, we are certainly glad that the special issue is
still going on.  We have hopefully addressed the remaining concerns raised
during the second round of reviews.  Besides, we have updated many bibliography
references (and added new ones) to reflect the continous progress of the field
since the initial submission in 2013.

\subsection*{Language}

Your first and most stringent comment relates to the language. We have carefully
proof-read the manuscript with the help of a native English speaker.
Specifically:

\begin{itemize}
    \item a special care has been given to rephrase colloquial expressions and
        non-academic phrasing,
    \item the abstract has been entierly rewritten,
    \item all the specific language changes pointed out by the editors and
        reviewers have been integrated.
\end{itemize}

To facilitate the review process, we attach an annotated version of the
manuscript where all the changes since the last submission have been highlighted.

\subsection*{Clarifiying the contribution}

We initially made the choice to discuss the existing litterature in relation to
our contribution in each of the ``cognitive skills'' sections, for readability
reasons. We however acknowledge that a higher-level discussion of our work with
respect to the state of the art of human-robot interaction systems would be
helpful.

To this end, we have:

\begin{itemize}
    \item added a summary of the key contribution at the end of the introduction
        (Section 1.2)
    \item split Section 5 into a discussion (Section 5) and a
conclusion (Section 6) that clearly states our contribution (including, per your
request, references to all our previous publications -- Section 6.1) and relate
it to the start of the art.
\end{itemize}

\subsection*{Reviewer Comments}

Reviewer 2 suggested that the discussion carried in section 5.3 "Knowledge and
Robotics" was out-of-scope for this article. After a light re-organization,
we have eventually decided to maintain it as we believe it is helpful in
pointing some of the limits of our approach.

The other comments from Reviewer 2 have all been addressed in the manuscript.
We highlight hereafter the main changes.

\begin{itemize}
    \item the re-organization of section 5 into section 5 'Discussion' and
        section 6 'Conclusion' hopefully addresses the concern that the last part of the
        paper was relatively disconnected from the core of the article;
    \item examples have been added to Section 2.2 to make it easier to follow;
    \item we have clarified how the system manages discrepancies between OpenCyc
        terminology and human natural language by explicit manual labelling
        (Section 2.2.2);
    \item We have precised that our approach of transient geometric situation assessment
        prevents reasoning on the situation history (Section 3.2);
    \item We have clarified how the {\tt nextTo} spatial relation is computed;
    \item We have removed the paragraph on state probabilities (Section 3.2.2)
        as it referred to an unfinished research proposal;
    \item We have clarified how primitive action recognition is used in
        combination with situation assessment and task planning to monitor the
        human activities (Section 3.2.3);
    \item The implicit aspect of multi-modality during dialogue processing has
        been made clearer (Section 3.3.2).
\end{itemize}


Finally, as suggested by Reviewer 3, we have changed the title of the article
from ``Human-Robot Interaction: Tackling the AI Challenges'' to the more
specific ``Artificial Cognition for Social Human-Robot Interaction: An
Implementation''.


\bibliographystyle{abbrv}
\bibliography{laas_hri}



\end{document}
