\documentclass{article}
\usepackage[utf8]{inputenc}


\begin{document}

Dear Editors,

\vspace{2em}

Thank you for coming back to us, we are glad that the special issue is still
going on.

We have hopefully addressed the remaining concerns raised during the second
round of reviews.

\subsection*{Language}

Your first and most stringent comment relates to the language. We have carefully
proof-read the manuscript with the help of a native English speaker.
Specifically:

\begin{itemize}
    \item a special care has been given to rephrase colloquial expressions and
        non-academic phrasing,
    \item the abstract has been entierly rewritten,
    \item all the specific language changes pointed out by the editors and
        reviewers have been integrated.
\end{itemize}

To facilitate the review process, we attach an annotated version of the
manuscript where all these changes have been highlighted.

\subsection*{Relation to the state of the art}

We initially made the choice to discuss the existing litterature in each of the
``cognitive skills'' sections, for readability reasons. We however acknowledge
that a higher-level discussion of our work with respect to the state of the art
of human-robot interaction systems would be useful.

To this end, we have split Section 5 into a discussion (Section 5) and a
conclusion (Section 6) that states clearly our contribution (including
references to all our previous publications -- Section 6.1), followed by a
comparative discussion with the related approaches (Section 6.2).

\subsection*{Reviewer Comments}

Reviewer 2's comments have all been addressed in the manuscript.
We highlight hereafter the main changes.

\begin{itemize}
    \item examples have been added to Section 2.2 to make it easier to follow,
    \item we have clarified how the system manages discrepancies between OpenCyc
        terminology and human natural language by explicit manual labelling
        (Section 2.2.2)
    \item We have precised that our approach of transient geometric situation assessment
        prevents reasoning on the situation history (Section 3.2)
    \item We have clarified how the {\tt nextTo} spatial relation is computed.
    \item We have clarified how primitive action recognition is used in
        combination with situation assessment and task planning to monitor the
        human activities (Section 3.2.3)
    \item The implicit aspect of multi-modality during dialogue processing has
        been made clearer (Section 3.3.2)
\end{itemize}


Finally, as suggested by Reviewer 3, we have changed the title of the article
from ``Human-Robot Interaction: Tackling the AI Challenges'' to the more
specific ``Artificial Cognition for Social Human-Robot Interaction: An
Implementation''.


\bibliographystyle{abbrv}
\bibliography{laas_hri}



\end{document}
