\documentclass{letter}
\usepackage{letterbib}
\usepackage[utf8]{inputenc}
\usepackage{hyperref}
\usepackage{booktabs}


\signature{Séverin Lemaignan}
\begin{document}

\subsection{General Comments}

\emph{1. The novelty of this paper with respect to other papers by the same
authors, including one in IJSR 2011, has been questioned. The authors
should convincingly explain what are the novel contributions and the
added value of this work compared to the previous publications.}

We address this question with our reponse to reviewer 3, below.

\emph{2. The form and depth of the presentation are below the standards of an
AIJ article.  The reviewers made many suggestions to make the presentation
better structured, more detailed, and more focused.}

We detail below the steps that were taken to improve the overall quality of the
paper (note that minor typos/style issues reported by the reviewers have been
fixed directly in the manuscript, and are not discussed here anymore).


\subsection{Comments from Reviewer 1}

\begin{itemize}

    \item \emph{For longer journal articles, a citation style like 'apalike' that uses the
author names in the references would be more pleasant to read since it saves
reviewers from jumping back and forth all the time.

[EiC note: please use the standard AIJ referencing style -- we do not have the
liberty to change this -- thanks]}

As indicated by the Editor in Chief, we need to keep the current citation style.

    \item \emph{p10 l 49: How is this discrimination done?} The discrimination
        algorithms are detailed in \cite{Ros2010b} (Best Paper Award at
        ROMAN2010). We have added a reference to this paper.

    \item \emph{The example scenarios have very bold names for often rather
        simple things.} The reviewer is indeed right. We have mistakenly used internal
        \emph{code names} for the experiments' names. These have been reworded.
\end{itemize}

\subsection{Comments from Reviewer 2}

\begin{itemize}
    \item \emph{pg 10 (section 3.2.1)
            "The system supports ...". This is one example of a topic that would
            be really interesting to the readers and it is just hinted. In
            particular, learning rules like "cats are animals" maybe really
            nice, but are such rules compatible with the existing knowledge?
        Adding them as triples may work only in absence of contradictions} The
        various natural language grounding processes that are made possible
        within this semantic
        architecture would require a relatively long discussion to be adequately
        explained. They are besides already presented in~\cite{Lemaignan2011a}.
        We have added an explicit reference to this previous publication.

        To specifically answer the reviewer question on how possible
        contradictions are managed, we indeed check that new statements proposed
        by the human do not lead to contradictions before inserting them.

    \item \emph{pg 10 (section 3.2.1)

            "Note that, while the system benefits from the complementary
            modalities, they are not all
            required. The dialogue system can run with only the verbal modality,
            at the cost of a simpler interaction."
        Possibly true, but unjustified claim.} An example has been added to
        clarify this fall-back mechanism.

    \item \emph{"The second level of integration of multi-modality is implicit:
            by computing symbolic properties 
            from the geometry, richer descriptions and hence discrimination
            possibilities are available: 
            for instance, if reachability is available, the robot may ask "do
            you mean the bottle that is 
            accessible to me?" to discriminate between the three bottles. That
            way, procedures relying on 
            discrimination transparently benefit from added modalities." Concept
        remains unclear.} The paragraph has been rephrased and the example
        updated to clarify the concepts.
\end{itemize}


\end{document}
