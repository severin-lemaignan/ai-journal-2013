\documentclass{article}
\usepackage[utf8]{inputenc}


\begin{document}

\section{General Comments}

\begin{itemize}

    \item \emph{The form and depth of the presentation are below the standards of an
AIJ article.  The reviewers made many suggestions to make the presentation
better structured, more detailed, and more focused.}

We would like first to thank the reviewers for their thoughtful review of our
work. The manuscript underwent a major rewrite, impacting both the overall style
(which should now be much more pleasant to read) and the whole organisation of
the paper (following the reviewers' recommendations). We also dedicated time to
redraw the figures in an improved academic style.

We address hereafter each of the specific concerns raised by the reviewers (note
that minor typos/style issues have been fixed directly in the manuscript, and
are not discussed here).


\item \emph{The novelty of this paper with respect to other papers by the same
authors, including one in IJSR 2011, has been questioned. The authors should
convincingly explain what are the novel contributions and the added value of
this work compared to the previous publications.}

We address this question with our reponse to reviewer 3, below.

\end{itemize}

\section{Comments from Reviewer 1}

\begin{itemize}

    \item \emph{For longer journal articles, a citation style like 'apalike' that uses the
author names in the references would be more pleasant to read since it saves
reviewers from jumping back and forth all the time.}

As indicated by the Editor in Chief, we need to keep the current citation style.

    \item \emph{p9: Instead of storing these models separately (having to
    update each
of them), wouldn't it be easier to compute them once they are needed, and
to maintain a quasi-global world belief?} This is a design choice that trades
off scalability for complete domain explicitation. We have clarified and
justified this design choice in section 2.2 (\emph{Knowledge model}). Also, in
the specific case of agent-dependent models, we want to represent
false-beliefs situations (where one agent carries wrong beliefs on the world)
that can typically not be computed ``on demand'' because they are observable
only in specific situations.

    \item \emph{p10 l 49: How is this discrimination done?} The discrimination
        algorithms are detailed in \cite{Ros2010b} (Best Paper Award at
        ROMAN2010). We have added a reference to this paper.

    \item \emph{p15 While the production of such rules is one thing, managing the
    dynamics of this knowledge and e.g. removing it once it's outdated or updating
once it has changed are crucial. This should be explained in more detail.}
The dynamics of the knowledge are managed by the clients (ie, the deliberative
components like the situation assessment module, the dialogue processing module,
etc.) of the knowledge base themselves. In particular, the situation assessment
module recomputes at every step the whole set of spatial relations, and update
accordingly the knowledge base. This has been clarified in both section 2.2
(\emph{Knowledge model}) and section 3.2 (\emph{Situation Assessment, Human
Awareness, Perspective Taking})

    \item \emph{The example scenarios have very bold names for often rather
        simple things.} The reviewer is indeed right. We have mistakenly used internal
        \emph{code names} for the experiments' names. These have been reworded.

    \item \emph{p 20: AFAIK KnowRob stores OWL as facts in Prolog, why
        should the TBOX be immutable?} The reviewer is right insofar as there is
        not limitations inherent to Prolog in that regard. We have removed the
        reference to KnowRob TBOX, that originated in a discussion with KnowRob's author who told
        us then that KnowRob could not update at run-time its TBox. Since we were
        not able to find litterature confirming this fact, it is likely better
        not to mention it.

\end{itemize}

\section{Comments from Reviewer 2}

Reviewer 2 raised several concerns regarding the overall article organisation
and lack of focus: \emph{``The overall impression is that the paper tries to
cove a too brad set of topics. As a result the takeaway message for the user is
too vague.''}

S/he recommended to modify the section 3 to either provide more informations
\emph{``to understand the problems and the technical solutions''} or to focus on
one particular cognitive skill; asked for \emph{``more substantial validation
of the specific problem(s) chosen in section 3''} in section 4, and suggested
several modifications to the discussion.

We have attempted to address these concerns in the following way:

\begin{itemize}
    \item section 3 underwent a large rewrite, in particular to provide
        technical insights, as well as a better overall picture of the role and
        importance of these skills. We have also added a subsection on the
        reasoning processes that take place in the background.
    \item the section 4 (previously ``experimental outcomes'') has been
        rewritten to make clear that we only present here \emph{case studies}
        that are meant to illustrate and clarify how the whole architecture
        works ``in the real world''. Besides, more details have been provided to the
        \emph{Clean the Table} study (as well as a redesigned figure) to help
        the user to follow the execution flow.
    \item the discussion section has been entirely reorganised, following the
        Reviewer recommendation. The key decisional issues have been explicitly
        stated, and overlaps between subsections have been addressed.
\end{itemize}

We address hereafter the other specific points raised by the reviewer:

\begin{itemize}

    \item \emph{pg. 7 (Section 3.1) ``From an artificial intelligence
            perspective, Spark takes care of merging sensing modalities, it
            computes and grounds perceptions into symbolic knowledge (situation
            assessment), it is responsible temporal interpretation of situations
            and performs spatial perspective-taking.'' given on this
        characterization, Spark should be discussed in greater detail.} We have
        significantly developed the presentation of {\sc Spark}, in
        particular to exhaustively cover what and how symbolic relations are
        computed.

    \item \emph{pg 10 (section 3.2.1)
            "The system supports ...". This is one example of a topic that would
            be really interesting to the readers and it is just hinted. In
            particular, learning rules like "cats are animals" maybe really
            nice, but are such rules compatible with the existing knowledge?
            Adding them as triples may work only in absence of contradictions} The
            various natural language grounding processes that are made possible
            within this semantic
            architecture would require a relatively long discussion to be adequately
            explained. They are besides already presented in~\cite{Lemaignan2011a}.
            We have added an explicit reference to this previous publication.

            To specifically answer the reviewer question on how possible
            contradictions are managed, we indeed check that new statements proposed
            by the human do not lead to contradictions before inserting them.

    \item \emph{pg 10 (section 3.2.1)
            "Note that, while the system benefits from the complementary
            modalities, they are not all
            required. The dialogue system can run with only the verbal modality,
            at the cost of a simpler interaction."
        Possibly true, but unjustified claim.} An example has been added to
        clarify this fall-back mechanism.

    \item \emph{"The second level of integration of multi-modality is implicit:
            by computing symbolic properties 
            from the geometry, richer descriptions and hence discrimination
            possibilities are available: 
            for instance, if reachability is available, the robot may ask "do
            you mean the bottle that is 
            accessible to me?" to discriminate between the three bottles. That
            way, procedures relying on 
            discrimination transparently benefit from added modalities." Concept
            remains unclear.} The paragraph has been rephrased and the example
            updated to clarify the concepts.

    \item \emph{In the current form, Section 2.3 seems outside the scope of the
            paper. The discussion on working memory is quite interesting, but
            since it is still to be developed in detail it maybe a topic for
            future perspectives. In that case it would be nice to read which AI
            techniques could help tackling the issue.} The presentation of
            ``internal cognitive skills'' has been extended, refined and integrated
            to section 3. Specifically, several litterature references have been
            added to the discussion on memory, both on the theoritical side and on
            existing implementations.

    \item \emph{``Subclasses (like Give, LookAt, etc.) are not asserted in the
            common-sense ontology, but are added by the execution controller (in
            link with the symbolic task planner) and the natural language
            processor based on what is actually performable and/or
            understandable by the robot at run-time'' This is very difficult to
            understand. Provide more details and, possibly, an example} The paragraph
            has been rephrased and illustrated to be easier to understand.

\end{itemize}

\section{Comments from Reviewer 3}

The main concern of Reviewer 3 relates to the lack of originality of this
contribution.

The reviewer references two previous publications by the authors. \emph{Grounding
the Interaction: Anchoring Situated Discourse in Everyday Human-Robot
Interaction} has indeed been published in 2011 in the \emph{International Journal of
Social Robotics}. However, the overlap with this article appears to be minimal. The
discussion on natural language grounding, which is the focus of the previously
published article, is only briefly summarized in the present article (section
3.3) and as far as we can tell, only one picture, used as an example, as been
reused.

The second publication, \emph{When the robot considers the human...} is supposed
to appear in the proceedings of the \emph{2011 International Symposium on
Robotics Research}, but has yet to be actually pusblished.  While this article
indeed presents our architecture in a way similar to our proposed contribution,
the fact that it belongs to the proceedings of a conference and that it has
actually not yet been published should make it mostly irrelevant.

The reviewer 3 also expresses concerns regarding our proposal not \emph{``helping
researchers to build upon each other's work''} because of the way we
\emph{``presents our design choices''}. We understand this remark has being
related to the comment by Reviewer 2 regarding us not presenting clearly
enough the key issues of human-robot interaction and consequently, the
challenges that HRI brings to artificial intelligence. As explained above, we
have hopefully addressed this concern in the current version of the paper.

Generally speaking, we would like to underline that, while we acknowledge that a
lot of material has been already published on the different sub-systems of
our architecture (as rightfully noted by Reviewer 1), this article give for the
first time a complete picture of the system. This offers, in our opinion, a
perspective on the state of the art in cognitive robotic architectures
for human-robot interaction that may be of interest for the AI and robotics
research communities.


\end{document}
